\section{RC Week 8}

\subsection{Working with class invariants}	
\begin{frame}{Invariants of a datatype}
We now take a look at the implementation side. From a implementation point of view, what's the difference between a list of integers, and a set of integers? 
\begin{itemize}
	\item Both of them probably are represented using an array.
	\item Both of them might need to keep track of current number of elements. 
\end{itemize} 
The major difference lies in the fact that when inserting an element, integer set must check whether the element already exists. It is the assumptions on the member that makes them different.

\vspace{0.05in}
Integer set has one more assumption than integer list. Each number in the array must be unique. Every operation of an integer set must start with this assumption, might broke it in the middle, but ends up with it.

\vspace{0.05in}
We will use your \texttt{IntSet} example in class for our discussion.
\end{frame}

\begin{frame}{Example of invariants}
\begin{columns}
	\column[]{.5\textwidth}
	
	\vspace{-.3in}
	\inputminted[fontsize=\small]{c++}{code/rc8intset/intset.h}
	
	Invariants:
	
	1) Numbers are arranged in \texttt{elts} sequentially from index zero.

	2) Number of elements in array equals to the value of \texttt{numElts}.
	
	3) Numbers are unique.
	\column[]{.6\textwidth}
	
	\vspace{-.25in}
	Guideline is:
	\begin{itemize}
		\item Methods  assume inv.s on entry.
		\item (Public) Methods must preserve inv.s on the moment of exit.
		\item Nobody cares what happens in the middle!
	\end{itemize}
	This way we safely claim invariants are always preserved for an class instance from an external point of view. 
	
	Analyze the methods:
	\begin{itemize}
		\item \texttt{insert} might broke all three.
		\item \texttt{remove} might broke 1) and 2).
		\item \texttt{const} methods will not break anything! One more advantage!
	\end{itemize}
\end{columns}
\end{frame}

\begin{frame}{Example of invariants: Implementing methods}
\begin{columns}
	\column[]{.5\textwidth}
	
	\vspace{-.3in}
	
	\inputminted[fontsize=\small,xleftmargin=1em, linenos]{c++}{code/rc8intset/intset.cpp}
	
	\column[]{.6\textwidth}
	
	\vspace{-.25in}
	
	Analyze the methods:
	\begin{itemize}
		\item L.9 checks inv.3 for uniqueness.
		\item L.7 checks inv.2 for capacity. If violation of inv cannot be recovered from, depending on how serious situation is you can \texttt{throw} or fail-fast.
		\item L.9 uses inv.1. We insert at the end, temporarily breaks invariance.
		\item L.9 operator \texttt{++} restores inv.
		\item Inv.3 ensures \texttt{remove} works.
		\item L.15 restores inv.1. L.16 restores inv.2. A significant amount of code is dedicated to preserving invs.
	\end{itemize}
\end{columns}
\end{frame}


\begin{frame}{Remarks on invariants}
We would like to note the following things:
\begin{center}
	\structure{The centric problem in choosing an implementation\\ for an ADT, is choosing the invariants.}
\end{center}
There are multiple ways to write an implementation an \texttt{IntSet} of course. The major difference of them, is they need to keep different invariants, and thus have different performance. 

For example, you have already seen \texttt{IntSet} implemented using a sorted array. The one extra invariant you need to keep is that elements are sorted. Based on that extra invariant:
\begin{itemize}
	\item Queries are much quicker.
	\item Deletion and insertion are slower (by a constant factor).
\end{itemize}
In VE281 you will see (many) other ways of implementing this IntSet, for example using hash tables, or balance trees. It is always helpful to understand what the invariant for an implementation.
\end{frame}

\begin{frame}{Invariants and bad practices}
Thinking about invariants immediately gives us hints on what are seemingly good, some even sounds terrific, but are actually pretty bad ideas. We now take a look at some of them.

\begin{description}[Alice]
	\small
	\item[Alice] I'd like to add a public method to \texttt{IntSet}, let's say \texttt{find()}, which takes an element, found it in the set, and returns an reference to that element. 
	\item [Bob] Why do you want to do that?
	\item [Alice] Well my code needs to modify elements in the array, a lot. This would significantly accelerate the process.
	\item [Bob] ...
	\item [Alice] Oh, I get it, it's a pretty bad idea. OK I have another idea, I want to add a method, \texttt{data()}, which returns the pointer to array, since my code needs to iterate through the set. 
	\item [Bob] ...
	\item [Alice] What if I return pointer to const?
	\item [Bob] ...
	\item [Alice] Well, I give up. It's not as easy as I thought.
\end{description}
\end{frame}

\begin{frame}{Invariants and bad practices}
Well Alice does have her point. Her requirements are very real and the solutions does solutions could work, but remember the words from \textit{Donald Knuth}:
\begin{quotation}
	\structure{Premature optimization is the root of all evil}.
\end{quotation}
Never trade correctness for performance in designing phase. 
\begin{itemize}
	\small
	\item 1st and 2nd change Alice proposed have the problem that the client could change the content of the internal array, without notifying the instance. There is by no means the instance can guarantee the new value won't violate uniqueness invariance.
	\item Now Alice proposes the third one. The \texttt{const} key word could prevent the element from being modified. But it exposes the implementation to outside. Now the fact that elements are stored in a linear array becomes part of the abstraction, you can no longer change that. The problem is, iterating through a set seems to be an important feature, we DO need that. Well, the correct solution is using iterators, you will know them later in this course.
\end{itemize} 
\end{frame}

\begin{frame}{Invariants and bad practices}
Well Alice does have her point. Her requirements are very real and the solutions does solutions could work, but remember the words from \textit{Donald Knuth}:
\begin{quotation}
	\structure{Premature optimization is the root of all evil}.
\end{quotation}
Never trade correctness for performance in designing phase. 
\begin{itemize}
	\small
	\item 1st and 2nd change Alice proposed have the problem that the client could change the content of the internal array, without notifying the instance. There is by no means the instance can guarantee the new value won't violate uniqueness invariance.
	\item Now Alice proposes the third one. The \texttt{const} key word could prevent the element from being modified. But it exposes the implementation to outside. Now the fact that elements are stored in a linear array becomes part of the abstraction, you can no longer change that. The problem is, iterating through a set seems to be an important feature, we DO need that. Well, the correct solution is using iterators, you will know them later in this course.
\end{itemize} 
\end{frame}

\begin{frame}{Invariants and the constructor}
\alert{Invariants are there as long as the object is there.} The earliest moment the object is created, is when it is defined. Classes, are just structures, in terms of memory. Structures contains undefined values when created, so naturally does the classes. 

\vspace{0.03in}
Think about \texttt{IntSet}, when the object is created, chances are \texttt{numElts} contains non-zero value, while your set should be empty.

\vspace{0.03in}
We need a mechanism, a function, that is invoked whenever the an instance is created. This function is called, and called always, to initialize the object, or more strictly, to setup the invariance.

\vspace{0.03in}
Such special functions are \alert{constructors}, or \alert{ctors} for short. From this point on it's important to think the initialization, not as assigning special values to member variables, but as invoking the constructor to setup invariants (together with the initial state, e.g. initialize the \texttt{IntSet} with one default element in it for whatever reason).
\end{frame}

\begin{frame}[fragile]{Constructors: Syntax}

\begin{columns}
	\column[]{.5\textwidth}
	
	\vspace{-.3in}
	The construct for \texttt{IntSet}:
	\inputminted[]{c++}{code/rc8intset/intset_ctor.h}
	
	
	\column[]{.6\textwidth}
	
	\vspace{-.65in}
	
	\begin{itemize}
		\small
		\item Ctors have same name as class.
		\item Ctors don't return anything. 
		\item Usually \texttt{public}, but could be \texttt{private} for special need.
		\item A class could have more than one ctor, depending on what the initial state should be. We say the constructor is overloaded.
		\item Ctors could take initialization list.
		\item Ctor is the first function being called when the object is created. Unfortunately ``created" is not as simple as it sounds.
		\item A ctor that does not take argument is called \texttt{a default ctor}. Every class must always have a ctor. If you don't specify \alert{any ctor}, a ctor is synthesized for you (under some conditions).
	\end{itemize}
\end{columns}
\tiny{* We left out copy/move constructors on purpose. You will learn them later.}
\end{frame}

\begin{frame}[fragile]{Constructors: Explanations}
We need to break down the words from previous slide. We first look at the first four lines. The first three seems direct. For the fourth one:
\begin{minted}{c++}
IntSet set1; // set1 is init to an empty set;
IntSet set2(10); // set2 contains single elem 10
\end{minted}

\begin{minted}{c++}
int foo(const IntSet& s1, const IntSet& s2);
foo(IntSet(), IntSet(10)); // Anymous construction
\end{minted}

Further more, latest C++ standard guarantees the following:
\begin{minted}{c++}
IntSet setx = IntSet();    // Same as set1 
IntSet sety = IntSet(10);  // Same as set2
IntSet setz = 10;          // Same as set2
\end{minted}
This is non-trivial, for reasons you will see one or two weeks later. This is called \textit{guaranteed copy elision}. Now some food for thought: 

Why \texttt{set1} is not initialized as \texttt{IntSet set1();} for consistency?
\end{frame}

\begin{frame}[fragile]{Constructors: Explanations}
We now question: Is constructor really the first function being called when created? Well that depends on how you understand ``created"! One should argue the ctor IS part of object creation.

\vspace{0.1in}
In fact, there are 4 steps (roughly) in creating an class instance.
\begin{enumerate}
	\item Allocation of space, we (in general) don't have control over.
	\item Construction of the base class object.
	\item Construction of the members of current class.
	\item Calling the constructor.
\end{enumerate}
These 4 steps goes recursively. In fact the constructor is the last function being called. This is natural. 

When the constructor is called, we should be able to use all member variables and base class. So their invariants must have been setup in advance! This is the only way things makes sense.
\end{frame}

\begin{frame}[fragile]{Initializer list: Setup}
The \textit{initializer list}, the things after colon controls the second and third step. 

\begin{minted}{c++}
ClsName::ClsName() : base(..), m1(..), m2(..) {
    // Code for the constructor goes here
}
\end{minted}

%You might wonder, wait a minute, I have never write initializer list before, I don't even write a constructor!

To better illustrate our ideas, we make some changes to our original \texttt{IntSet}, specifically 2 changes:

\begin{itemize}
	\item We add a line of output to the ctors to indicate with constructor is being called.
	\item We make 2 more ``version" of our original \texttt{IntSet}, one \texttt{OddIntSet} and \texttt{EvenIntSet}. They only allow odd and even numbers, respectively.
\end{itemize}

We now propose a slightly faster \texttt{IntSet}. This \texttt{IntSet} is composed of an \texttt{OddIntSet} and \texttt{EvenIntSet}. It dispatches the operation on integers to those two sub-\texttt{IntSet}. 
%
%Well, if you don't provide any initializer list, the compiler calls the default constructor on every ctor, if a default ctor is available. If the default ctor is not available, a compile error is thrown.
\end{frame}

\begin{frame}[fragile]{Initializer list: Setup}
The \textit{initializer list}, the things after colon controls the second and third step. 

\begin{minted}{c++}
ClsName::ClsName() : base(..), m1(..), m2(..) {
    // Code for the constructor goes here
}
\end{minted}



To better illustrate our ideas, we make some changes to our original \texttt{IntSet}, specifically 2 changes:

\begin{itemize}
	\item We add a line of output to the ctors to indicate with constructor is being called.
	\item We make 2 more ``version" of our original \texttt{IntSet}, one \texttt{OddIntSet} and \texttt{EvenIntSet}. They only allow odd and even numbers, respectively.
\end{itemize}

We now propose a slightly faster \texttt{IntSet}. This \texttt{IntSet} is composed of an \texttt{OddIntSet} and \texttt{EvenIntSet}. It dispatches the operation on integers to those two sub-\texttt{IntSet}. 
%

\end{frame}

\begin{frame}[fragile]{Initializer list: Setup II}
\begin{columns}
	\column[]{.5\textwidth}
	
	\vspace{-.3in}
	Decl. of \texttt{OddIntSet}
	\inputminted[fontsize=\small]{c++}{code/rc8init/oddintset.h}
	
	\column[]{.5\textwidth}
	
	\vspace{-.3in}
	Partial implementation:
	\inputminted[fontsize=\small]{c++}{code/rc8init/oddintset.cpp}
\end{columns}

We omit the very similar version for \texttt{EvenIntSet}.
\end{frame}

\begin{frame}[fragile]{Initializer list: Setup III}
\begin{columns}
	\column[]{.5\textwidth}
	
	\vspace{-.2in}
	Declaration of \texttt{OddIntSet}
	\inputminted[fontsize=\small]{c++}{code/rc8init/fastintset.h}
	
	\column[]{.5\textwidth}
	
	\vspace{-.2in}
	Partial implementation:
	\inputminted[fontsize=\small]{c++}{code/rc8init/fastintset.cpp}
\end{columns}

\end{frame}

\begin{frame}[fragile]{Initializer list: Example}
We create a main function and instantiate an instance of \texttt{FastIntSet}. We observe the following ouput:
\begin{minted}{text}
OddIntSet dft ctor
EvenIntSet elt ctor
fIntSet ctor
\end{minted}
This implies an initialization order of \texttt{oddSet-evenSet-ctor}. This follows from our expectation. 

\vspace{0.1in}
A (not so important) remark we would like to make is
\begin{itemize}
	\item The order of initialization of the members is guaranteed by standard. It's NOT a UB!
	\item Contrary to intuition \textit{it is not specified by initializer list}, but by the order they appear in the declaration.
	\item Base objects are guaranteed to be initialized before members.
	\item We recommend you to specify the init-list in the same order as the declaration, for clearance.
\end{itemize}
\end{frame}

\begin{frame}[fragile]{Initializer list: Default initialization}
We now make a tiny change to the original setup. We change the ctor for \texttt{FastIntSet} to 
\begin{minted}{c++}
FastIntSet::FastIntSet() 
    : oddSet() {cout << "fIntSet ctor\n";}
\end{minted}

Oops, we forgot to specify initializer for \texttt{evenSet}.

\begin{minted}{text}
OddIntSet dft ctor
EvenIntSet dft ctor
fIntSet ctor
\end{minted}

\texttt{evenSet} is still initialized, no surprise, otherwise the invariant for \texttt{oddSet} would be broken. Members are default constructed, if they don't appear in the initializer list. We could even simply write 

\begin{minted}{c++}
FastIntSet::FastIntSet() {cout << "fIntSet ctor\n";}
\end{minted}

Which will produce the same result. Now what about members like \texttt{int}? Well, you can think of them as a class, whose default constructor does nothing at all!
\end{frame}

\begin{frame}[fragile]{Initializer list: Default initialization}
We can even leave out the constructor for \texttt{FastIntSet} entirely. In which case we would have the following output.

\begin{minted}{text}
OddIntSet dft ctor
EvenIntSet dft ctor
\end{minted}

Essentially the compiler would synthesize (create) and constructor for you, whose initializer list initializes everything using their default constructor.  

You might wonder, wait a minute, I have never write initializer list before, I don't even write a constructor before this lesson! That's how you ``get away" with writing constructors before!

Modern design principal is against leaving things unspoken. If you believe you only need a default constructor, you could do:

\begin{minted}{c++}
IntSet::InSet() = default; // In class decl.
\end{minted}

Next up we will look at something that is not taught in class, but problems you will meet in your own struggle with C++.
\end{frame}

\begin{frame}[fragile]{Initializer list: Default initialization}
We first change the ctor of \texttt{FastIntSet} to 
\begin{minted}{c++}
FastIntSet::FastIntSet() 
    : oddSet() {cout << "fIntSet ctor\n";}
\end{minted}

Then we \textbf{remove the default ctor} for \texttt{EvenIntSet}.

We compile the program. The question is what SHOULD happen?

\pause
\vspace{0.1in}
From a design point of view, if a class does not provide a default constructor, indicates the class is not (more likely should not be) default constructible, there is no way to setup a invariance without supplying an argument. Think about it, what should be the default skin color for an instance of a human class? None.

In this case, any reasonable design would require an compile error is thrown. This is exactly the case. They compiler will look for \texttt{evenIntSet::evenIntSet()} and reports a not-found error. \alert{Situation is similar if you leave out the entire ctor of \texttt{FastIntSet}.}
\end{frame}

\begin{frame}{Why initializing in initialization list is better?}
You are asked this question in the lecture slides but the question is left unanswered in the lecture slides.

\vspace{0.1in}
Frankly the questions really should be the other way around: on what grounds do initialization in the code is better? I can think of the following reasons you should prefer (in some cases you have to use) initialization list:
\begin{itemize}
	\item Performance. This will be discussed later on. In some cases the performance could be significant.
	\item Exception safety. We will discuss this when we you learn the rule of big three (five, actually).
	\item A member that don't have a default constructor must be initialized in the initialization list.
	\item \texttt{const} members and references can only be initialized in the initialization list.
	\item Semantic reasons.
\end{itemize} 
\end{frame}

\begin{frame}[fragile]{Comments on the performance argument}
A final comment on the performance argument. Suppose we write 
\begin{minted}{c++}
FastIntSet::FastIntSet() {
  cout << "fIntSet ctor\n";
  oddSet = OddIntSet(); evenSet = eventSet(10);
}
\end{minted}
You will observer 5 lines of output:
\begin{minted}[fontsize=\small]{text}
OddIntSet dft ctor // EvenIntSet dft ctor //
fIntSet ctor // OddIntSet dft ctor // EvenIntSet elt ctor //
\end{minted}
Things that actually happened (with significant cost) are :
\begin{itemize}
	\item Your member variables are first default constructed.
	\item Calls your constructor.
	\item You create 2 temporal objects, by default construction.
	\item You invoke an assignment operator and uses your temporal object to overwrite the member variables. 
\end{itemize}

\end{frame}